\documentclass[a4paper]{article}

\usepackage[english]{babel}
\usepackage[utf8]{inputenc}
\usepackage{amsmath}
\usepackage{graphicx}
\usepackage{color}
\usepackage[colorlinks,linkcolor=blue]{hyperref}
\usepackage[colorinlistoftodos]{todonotes}
\usepackage[noindent]{ctex}

\title{\Huge \heiti{操作系统实验(一)问答题参考}}

\author{\lishu{南京大学软件学院}}

\date{\normalsize 2015.4}
\begin{document}
\maketitle

\renewcommand{\abstractname}{实验重点}

\begin{abstract}
本次作业重点在于熟悉掌握:8086寻址方式和指令系统,主程序和子程序的参数传递以及$nasm+bochs$实验平台的搭建和使用
\end{abstract}




\section{问题清单}
~~~~~~在整个实验的过程中,无论是编程还是查资料,请各位同学注意思考以下问题,助教检查时会从中随机抽取数个题目进行提问,根据现场作答给出分数。
\underline{请注意,我们鼓励自己思考和动手实验},如果
能够提供自己的思考结果并辅助以相应的实验结果进行说明,在分数评定上会酌情考虑。

\begin{enumerate}
	\item boot.asm文件中,\textbf{org~~0700h}的作用
	
	参考答案:
	
	~~~~~~告诉汇编器该段代码会被加载到内存的07c00处,当编译的时候遇到相对寻址的指令的时候会用07c00加上相对地址得到绝对地址,
	\item 为什么要把boot.bin放在第一个扇区?直接复制为什么不行?
	
	参考答案:
	
	~~~~~~BIOS程序检查软盘0面0磁道1扇区,如果扇区以0xaa55结束,则认定为引导扇区,将其512字节的数据加载到内存的07c00处,然后设置PC,跳到内存07c00处开始执行代码。
	
~~~~~~普通的读写操作(mv, rm,cp)是基于文件系统的,文件系统是一个逻辑概念。而引导扇区,是磁盘第一个磁道的第一个扇区,他是一个物理概念,在文件系统中,这个扇区是不可见的。
	\item loader的作用有哪些?
	
	参考答案:
	
	~~~~~~加载内核入内存,跳入保护模式,内存分页。
	
	\item 说明下面每行代码的意思。(由于检查时发现问题,这道题目发生了小的变化)

\begin{table}[hbp]
\centering
\begin{tabular}{cc}
行号 & 代码 \\\hline
1 & L1 db 0\\
2 & L6 dd 1A92h\\
3 & mov al, [L1] \\
4 & mov eax, L1 \\
5 & mov [l1], ah \\
6 & mov eax, [L6] \\
7 & add eax, [L6] \\
8 & add [L6], eax \\
9 & mov al, [L6]
\end{tabular}
\end{table}

参考答案:

~~~~~~L1 db 0; 字节变量L1,初始值为0

~~~~~~L6 dd 1A92h ; 双字变量初始化成十六进制1A92

~~~~~~3 mov al, [L1] ; 复制L1里的字节数据到AL

~~~~~~4 mov eax, L1 ; EAX = 字节变量L1代表的地址

~~~~~~5 mov [L1], ah ; 把AH拷贝到字节变量L1

~~~~~~6 mov eax, [L6] ; 复制L6里的双字数据到EAX

~~~~~~7 add eax, [L6] ; EAX = EAX + L6里的双字数据

~~~~~~8 add [L6], eax ; L6 = L6里的双字数据+ EAX

~~~~~~9 mov al, [L6] ; 拷贝L6里的数据的第一个字节到AL

注:

~~~~~~word(字) 2个字节

~~~~~~double word(双字) 4个字节

~~~~~~quad word(四字) 8个字节

~~~~~~paragraph(一节) 16个字节
		
	\item \textbf{times 510-(\$-\$\$) db 0}
		
		为什么是510? \$和\$\$分别表示什么?不用times指令怎么写(等价命令)?
		
		参考答案:
		
		~~~~~~因为需要填充512个字节的数据,最后两个字节是以0xaa55结尾,所以需要填充510个字节\$表示当前的字节数,\$\$表示开始的字节。
		
~~~~~~不用times命令可以使用 db 0 循环(\$-\$\$)次
		
	\item 解释db命令:\underline{\textbf{L10 db “w”, “o”, “r”, “d”, 0  }}这条语句的意义,并且说明数字0的作用。
	
	参考答案:
	
	~~~~~~填充字符串“word”,最后的0表示结束符,即在C/C++里字符串末尾的结束符。
	
	备注:
	
	~~~~~~这个答案也是大多数同学的答案,可以说正确,也可以说不正确,因为是依赖于场景的,这道题目来源于参考资料PCASM一书,那本书中调用了C库函数,所以注释中是上文答案。而0其实并不是汇编语言中的结束符,汇编里面的结束符是\$等符号(根据平台有所不同)表示。(感谢131250012同学提出这个问题,并写程序实验验证,他将得到额外的加分,如果其他同学在以后的实验中有类似性质的实验验证,也将得到奖励)。
	
	\item \textbf{L1 db 0}
		
		\textbf{L2 dw 1000}
		
		L1、L2是连续存储的吗?即是否L2就存储在L1之后?
		
		参考答案:连续定义的数据储存在连续的内存中。也就是说,字L2就储存在L1的后面。
		
				
	\item 要是不知道L6标识的是多大的数据,下面这句话对不对?
		
		\textbf{mov [L6], 1}
		
		参考答案:
		
		~~~~~~This statement produces an operation size not specified error. Why?
Because the assembler does not know whether to store the 1 as a byte, word
or double word. To fix this, add a size specifier:
mov dword [L6], 1 ; store a 1 at L6
This tells the assembler to store an 1 at the double word that starts at L6.
Other size specifiers are: BYTE, WORD, QWORD and TWORD.
	\item 如何处理输入输出?在代码中哪里体现出来?
	
	参考答案:
	
	~~~~~~使用中断处理。
	\item 通过什么来保存前一次的运算结果?在代码中哪里体现出来?
	
	参考答案:
	
	~~~~~~栈或者寄存器
	\item 随机选择代码段,说明作用。
	\item 有哪些段寄存器?
	
	参考答案:
	
	~~~~~~代码段,数据段,堆栈段,附加段
	\item 8086/8088存储单元的物理地址长是多少?地址总线有多少位?可以直接寻址的物理空间是多少?
	
	参考答案:
	
	~~~~~~分别是20、20、1M。
	
	~~~~~~说明:
8086/8088的地址总线都是20位的;
外部数据总线宽度:8086:16位;8088:8位;
内部数据总线宽度相同,都是16位
	
	参考答案:
	
	~~~~~~物理地址=段值*16+偏移(左移四位)(实模式下如此,其他模式下16会变化)
	\item 寄存器的寻址方式(知道如何计算)。
	
	参考答案:
	
	~~~~~~立即寻址方式;寄存器寻址方式;直接寻址方式;寄存器间接寻址方式;寄存器相对寻址方式;基址加变址寻址方式;相对基址加变址寻址方式
	\item 几个常用指令的作用(如MOV,LEA等)。
	
	参考答案:
	
	~~~~~~MOV:把一个字或字节从源操作数SRC送至目的操作数DST
	
~~~~~~LEA:把操作数OPRD的有效地址传送到操作数REG

~~~~~~PUSH、POP:堆栈操作指令

~~~~~~ADD、ADC、SUB、SBB:加减运算指令

~~~~~~——其他见PPT,更详细的请翻阅《80X86》
	\item 主程序与子程序的几种参数传递方式。
	
	参考答案:
	
	~~~~~~利用寄存器传递参数
	
~~~~~~利用约定存储单元传递参数

~~~~~~利用堆栈传递参数

~~~~~~利用CALL后续区传递参数
	
\end{enumerate}

\section{参考资料}
	\begin{enumerate}
		\item 《Orange'S:一个操作系统的实现》
		\item \href{http://www.nasm.us/doc/} {NASM doc}
		\item \href{http://jingliu.me/my_files/nasm.pdf}{Introduction to NASM}
		\item \href{http://www.psut.edu.jo/sites/qaralleh/uplab/doc/MASM_Tutorial.pdf}{MASM Tutorial}
	\end{enumerate}


%\begin{thebibliography}{9}
%\bibitem{nano3}
%  K. Grove-Rasmussen og Jesper Nygård,
%  \emph{Kvantefænomener i Nanosystemer}.
%  Niels Bohr Institute \& Nano-Science Center, Københavns Universitet

%\end{thebibliography}
\end{document}
