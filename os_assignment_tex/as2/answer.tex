\documentclass[a4paper]{article}

\usepackage[english]{babel}
\usepackage[utf8]{inputenc}
\usepackage{amsmath}
\usepackage{graphicx}
\usepackage[colorlinks,linkcolor=blue]{hyperref}
\usepackage[colorinlistoftodos]{todonotes}
\usepackage[noindent]{ctex}
\usepackage{amsmath}
\usepackage{lipsum}
\usepackage{color}


\title{\Huge \heiti{操作系统实验(二)问答题参考}}

\author{\lishu{南京大学软件学院}}

\date{\normalsize 2015.5}
\begin{document}
\maketitle

\renewcommand{\abstractname}{实验重点}

\begin{abstract}
本次作业重点:熟悉掌握Fat12文件系统,$gcc+nasm$联合编译实践以及了解实模式与保护模式的基本内容。
\end{abstract}

\section{问题清单}
~~~~~~在整个实验的过程中,无论是编程还是查资料,请各位同学注意思考以下问题,助教检查时会从中随机抽取数个题目进行提问,根据现场作答给出分数。
\underline{请注意,我们鼓励自己思考和动手实验},如果
能够提供自己的思考结果并辅助以相应的实验结果进行说明,在分数评定上会酌情考虑。

\subsection{PPT相关内容}
\begin{enumerate}
	\item 实模式下的寻址方式以及实模式的缺陷
	
	寻址方式:实模式下,段在内存中固定的位置(物理地址 = 段值*16 + 偏移 );
	
	缺陷:通过改变段寄存器的值,我们可以随心所欲的访问内存任何一个单元,而丝毫不受到限制,不能对内存访问加以限制,也就谈不上对系统的保护;内存中每个字节的地址能由不止一个的段基址加偏移表示,比如04808能够由047C:0048,047D:0038,047E:0028 or 047B:0058表示,分段地址之间的比较将复杂化。
	
	\item 保护模式下的寻址过程:
	\begin{itemize}
		\item 段寄存器中存储的是什么?GDT 是什么?LDT 是什么?如何区分 LDT 和 GDT? LDT 和 GDT 的区别是什么?如何定位到 Descriptor?Descriptor 的内容有哪些?
		\begin{itemize}
			\item 存储的是选择子,即描述符在描述符表中的位置
			\item GDT是全局描述符表;LDT是局部描述符表
			\item 存储的是选择子,即描述符在描述符表中的位置
			\item GDT是全局描述符表;LDT是局部描述符表
			\item 根据选择子中的第三位决定是GDT还是LDT区分
			\item DT(Local)与GDT相同,但是不是全局的,对于某个进程,它只知道它自己的LDT。每个进程有自己的LDT,访问自己的段时从LDT查询。进程从LDTR寄存器中获得LDT的位置,向它发起查询
			\item 如果是GDT,则根据GDTR和段寄存器中的内容定位到描述符
			\item 如果是LDT,则根据LDTR中的内容(选择子)和GDTR中的内容定位到描述符
			\item Descriptor中的内容包括是否在内存中,段的起始地址、界限、属性等内容
			\item GDTR中内容是全局描述符表的位置;LDTR中存储选择子,用于定位GDT中的某个LDT描述符,得到LDT的地址
			\item LDT存放在GDT中的原因是GDT表只有一个,是固定的
			\item 而LDT表每个任务就可以有一个,因此有多个,并且由于任务的个数在不断变化其数量也在不断变化
		\end{itemize}
		\item GDTR 中的内容是什么?LDTR 中存储的是什么?为什么 LDT 要放在 GDT 中?
		\begin{itemize}
			\item 全局描述符表的位置
			\item 选择子,用于定位GDT中的某个LDT描述符,得到LDT的地址
			\item GDT表只有一个,是固定的;而LDT表每个任务就可以有一个,因此有多个,并且由于任务的个数在不断变化其数量也在不断变化。如果只有一个LDTR寄存器显然不能满足多个LDT的要求。
		\end{itemize}
	\end{itemize}
	
	\item 选择子的作用:
		\begin{itemize}
			\item 选择子是什么?它的值存放在哪里?
			
			描述符在描述符表中的相对偏移,值存放在段寄存器里。
			\item 选择子里面的内容有哪些?
			
			选择子是一个2字节的数,共16位,最低2位表示RPL(请求特权等级),第3位表示查表是利用GDT(全局描述符表)还是LDT(局部描述符表)进行,最高13位给出了 所需的描述符在描述符表中的地址.
			\item 为什么偏移地址大小是13位?
			
			GDTR是一个48位的寄存器,其中32位表示段地址,16位表示段限(最大 64K,每个描述符8字节,故最多有64K/8=8K个描述符)
13位正好足够寻址8K项。
		\end{itemize}
	\item 描述符的作用:
	
	描述一个段是否在内存中,段的起始地址、界限、属性等内容
	\item GDTR/LDTR 的作用:
		\begin{itemize}
			\item GDTR的内容是什么?
			
			全局描述符表的位置
			\item LDTR的内容是什么?
			
			LDT的描述符在GDT中的相对偏移,根据GDTR中的内容即可得到LDT的描述符
		\end{itemize}
	\item 根目录区大小一定么?扇区号是多少?为什么?
	
	不一定  19  1(引导扇区)+9(FAT1)+9(FAT2) =19
	\item 数据区第一个簇号是多少?为什么?
	
	第一个簇号为2,在1.44M软盘上,FAT前三个字节的值必须是固定的,分别是0xF0、0xFF、0xFF,用于表示这是一个应用在1.44M软盘上的FAT12文件系统。本来序号为0和1的FAT表项应该对应于簇0和簇1,但是由于这两个表项被设置成了固定值,簇0和簇1就没有存在的意义了,所以数据区就起始于簇2。
	\item FAT 表的作用?
	
	记录硬盘中有关文件如何被分散存储在不同扇区的信息(也可以回答为了找到所有的簇(扇区))
	\item 解释静态链接的过程。
	
	相似段合并;重定位。	
	\item 解释动态链接的过程。
	
	动态链接器自举;装载共享对象 ;重定位和初始化 
	\item 静态链接相关PPT中为什么使用ld链接而不是gcc。
	
	使用ld进行连接的原因是为了避免gcc进行glibc的链接
	\item linux下可执行文件的虚拟地址空间默认从哪里开始分配。
	
	linux下可执行文件的虚拟空间地址默认从0x08048000开始分配
\end{enumerate}

\subsection{实验相关内容}
\begin{enumerate}
	\item BPB 指定字段的含义
	\item 如何进入子目录并输出(说明方法调用)
	\item 如何获得指定文件的内容,即如何获得数据区的内容(比如使用指针等)
	\item 如何进行 C 代码和汇编之间的参数传递和返回值传递
	\item 汇编代码中对 I/O 的处理方式,说明指定寄存器所存值的含义
	\item 可以要求解释某些看不懂的代码(我看不懂的话,你得讲给我听)
\end{enumerate}

\section{参考资料}
	\begin{enumerate}
		\item 《Orange'S:一个操作系统的实现
		\item \href{http://jingliu.me/my_files/nasm.pdf}{Introduction to NASM}
		\item \href{http://www.eit.lth.se/fileadmin/eit/courses/eitn50/Projekt1/FAT12Description.pdf}
		{An overview of FAT12}
		\item \href{http://www.iecc.com/linker/linker10.html}{Dynamic Linking and Loading}
	\end{enumerate}

\end{document}
